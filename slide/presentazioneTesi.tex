\documentclass{beamer}

\usetheme{Padova}

\title{Componente di controllo per l'elaborazione in tempo reale di flussi di dati aggregati}
%\subtitle{Curabitur sit amet mi magna}
\author{Niccolò Mantovani}
\date{30 Settembre 2021}


\begin{document}

	\maketitle

	\begin{frame}{Outline}
		\tableofcontents
	\end{frame}


	\section{L'azienda}

	\begin{frame}{L'azienda}

		\textbf{Datasoil s.r.l.} \vspace{.2em}
		
		\textit{Startup} Innovativa fondata nel 2016 come \textit{Open Innovation} nell’ambito delle aziende manifatturiere. \vspace{.2em}
		
		\begin{itemize}
			\item \textit{Software as a Service} in \textit{cloud} \textbf{B2B} e \textbf{B2C} \vspace{.5em}
			\item Analisi, elaborazione e fruizione di dati \vspace{.5em}
			\item Piattaforma proprietaria \textbf{SYN}
		\end{itemize}
		
		\begin{figure}[!h] 
    		\centering 
    		\includegraphics[width=0.2\columnwidth]{../immagini/ds_logo.png}
		\end{figure}

	\end{frame}

	\section{Il progetto}

	\begin{frame}{Il progetto}
		\begin{block}{Normal block}
			Fusce luctus venenatis felis quis semper
		\end{block}

		\begin{alertblock}{Alert block}
			$$ E = (x_1 \vee \neg x_2 \vee \neg x_3) \wedge (x_1 \vee x_2 \vee x_4) $$
		\end{alertblock}

		\begin{exampleblock}{Example block}
			Proin tincidunt, neque at tincidunt mollis
		\end{exampleblock}
	\end{frame}
	
	\subsection{Problema affrontato}

	\begin{frame}{Problema affrontato}
		\begin{block}{Normal block}
			Fusce luctus venenatis felis quis semper
		\end{block}

		\begin{alertblock}{Alert block}
			$$ E = (x_1 \vee \neg x_2 \vee \neg x_3) \wedge (x_1 \vee x_2 \vee x_4) $$
		\end{alertblock}

		\begin{exampleblock}{Example block}
			Proin tincidunt, neque at tincidunt mollis
		\end{exampleblock}
	\end{frame}
	
	\subsection{Soluzione proposta}

	\begin{frame}{Soluzione proposta}
		\begin{block}{Normal block}
			Fusce luctus venenatis felis quis semper
		\end{block}

		\begin{alertblock}{Alert block}
			$$ E = (x_1 \vee \neg x_2 \vee \neg x_3) \wedge (x_1 \vee x_2 \vee x_4) $$
		\end{alertblock}

		\begin{exampleblock}{Example block}
			Proin tincidunt, neque at tincidunt mollis
		\end{exampleblock}
	\end{frame}
	
	\section{Implementazione}

	\begin{frame}{Implementazione}
		\begin{block}{Normal block}
			Fusce luctus venenatis felis quis semper
		\end{block}

		\begin{alertblock}{Alert block}
			$$ E = (x_1 \vee \neg x_2 \vee \neg x_3) \wedge (x_1 \vee x_2 \vee x_4) $$
		\end{alertblock}

		\begin{exampleblock}{Example block}
			Proin tincidunt, neque at tincidunt mollis
		\end{exampleblock}
	\end{frame}
	
	\section{Strumenti utilizzati}

	\begin{frame}{Strumenti utilizzati}
		\begin{block}{Normal block}
			Fusce luctus venenatis felis quis semper
		\end{block}

		\begin{alertblock}{Alert block}
			$$ E = (x_1 \vee \neg x_2 \vee \neg x_3) \wedge (x_1 \vee x_2 \vee x_4) $$
		\end{alertblock}

		\begin{exampleblock}{Example block}
			Proin tincidunt, neque at tincidunt mollis
		\end{exampleblock}
	\end{frame}
	
	\section{Prodotto finale}

	\begin{frame}{Prodotto finale}
		\begin{block}{Normal block}
			Fusce luctus venenatis felis quis semper
		\end{block}

		\begin{alertblock}{Alert block}
			$$ E = (x_1 \vee \neg x_2 \vee \neg x_3) \wedge (x_1 \vee x_2 \vee x_4) $$
		\end{alertblock}

		\begin{exampleblock}{Example block}
			Proin tincidunt, neque at tincidunt mollis
		\end{exampleblock}
	\end{frame}
	
	\section{Considerazioni}

	\begin{frame}{Considerazioni}
		\begin{block}{Normal block}
			Fusce luctus venenatis felis quis semper
		\end{block}

		\begin{alertblock}{Alert block}
			$$ E = (x_1 \vee \neg x_2 \vee \neg x_3) \wedge (x_1 \vee x_2 \vee x_4) $$
		\end{alertblock}

		\begin{exampleblock}{Example block}
			Proin tincidunt, neque at tincidunt mollis
		\end{exampleblock}
	\end{frame}
	
	\subsection{Possibili miglioramenti}

	\begin{frame}{Considerazioni}
		\begin{block}{Normal block}
			Fusce luctus venenatis felis quis semper
		\end{block}

		\begin{alertblock}{Alert block}
			$$ E = (x_1 \vee \neg x_2 \vee \neg x_3) \wedge (x_1 \vee x_2 \vee x_4) $$
		\end{alertblock}

		\begin{exampleblock}{Example block}
			Proin tincidunt, neque at tincidunt mollis
		\end{exampleblock}
	\end{frame}
	
	\subsection{Conoscenze acquisite}

	\begin{frame}{Considerazioni}
		\begin{block}{Normal block}
			Fusce luctus venenatis felis quis semper
		\end{block}

		\begin{alertblock}{Alert block}
			$$ E = (x_1 \vee \neg x_2 \vee \neg x_3) \wedge (x_1 \vee x_2 \vee x_4) $$
		\end{alertblock}

		\begin{exampleblock}{Example block}
			Proin tincidunt, neque at tincidunt mollis
		\end{exampleblock}
	\end{frame}


\end{document}
