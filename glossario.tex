
%**************************************************************
% Acronimi
%**************************************************************
\renewcommand{\acronymname}{Acronimi e abbreviazioni}

\newacronym[description={\glslink{apig}{Application Program Interface}}]
    {api}{API}{Application Program Interface}

\newacronym[description={\glslink{umlg}{Unified Modeling Language}}]
    {uml}{UML}{Unified Modeling Language}
    
\newacronym[description={\glslink{sqlg}{Structured Query Language}}]
    {sql}{SQL}{Structured Query Language}
    
\newacronym[description={\glslink{pojog}{Plain Old Java Object}}]
    {pojo}{POJO}{Plain Old Java Object}

%**************************************************************
% Glossario
%**************************************************************
%\renewcommand{\glossaryname}{Glossario}

\newglossaryentry{apig}
{
    name=\glslink{api}{API},
    text=Application Program Interface,
    sort=api,
    description={in informatica con il termine \emph{Application Programming Interface API} (ing. interfaccia di programmazione di un'applicazione) si indica ogni insieme di procedure disponibili al programmatore, di solito raggruppate a formare un set di strumenti specifici per l'espletamento di un determinato compito all'interno di un certo programma. La finalità è ottenere un'astrazione, di solito tra l'hardware e il programmatore o tra software a basso e quello ad alto livello semplificando così il lavoro di programmazione}
}

\newglossaryentry{umlg}
{
    name=\glslink{uml}{UML},
    text=UML,
    sort=uml,
    description={in ingegneria del software \emph{UML, Unified Modeling Language} (ing. linguaggio di modellazione unificato) è un linguaggio di modellazione e specifica basato sul paradigma object-oriented. L'\emph{UML} svolge un'importantissima funzione di ``lingua franca'' nella comunità della progettazione e programmazione a oggetti. Gran parte della letteratura di settore usa tale linguaggio per descrivere soluzioni analitiche e progettuali in modo sintetico e comprensibile a un vasto pubblico}
}

\newglossaryentry{stateful}
{
    name=\glslink{stateful}{STATEFUL},
    text=Stateful,
    sort=stateful,
    description={In ambito informatico, \emph{stateful} significa che un applicativo supporta stati diversi. Cioè, dato lo stesso input, l'output dipenderà dallo stato corrente}
}

\newglossaryentry{stateless}
{
    name=\glslink{stateless}{STATELESS},
    text=Stateless,
    sort=stateless,
    description={In ambito informatico, \emph{stateless} significa che un applicativo non supporta stati diversi. Cioè, dato lo stesso input, l'output dipenderà solamente da esso}
}

\newglossaryentry{bounded streams}
{
    name=\glslink{bounded streams}{BOUNDED STREAMS},
    text=Bounded streams,
    sort=bounded streams,
    description={In Apache Flink, un \textit{Bounded streams} ( letteralmente \textit{flusso limitato}), rappresenta un flusso che ha un inizio ed una fine}
}

\newglossaryentry{unbounded streams}
{
    name=\glslink{unbounded streams}{UNBOUNDED STREAMS},
    text=Unbounded streams,
    sort=unbounded streams,
    description={In Apache Flink, un \textit{Unbounded streams} ( letteralmente \textit{flusso illimitato}), rappresenta un flusso che ha un inizio ma non una fine}
}

\newglossaryentry{framework}
{
    name=\glslink{framework}{FRAMEWORK},
    text=Framework,
    sort=framework,
    description={Un \textit{framework}, termine della lingua inglese che può essere tradotto come struttura o quadro strutturale, in informatica e specificamente nello sviluppo software, è un'architettura logica di supporto (spesso un'implementazione logica di un particolare design pattern) sulla quale un software può essere progettato e realizzato, spesso facilitandone lo sviluppo da parte del programmatore}
}


\newglossaryentry{serializzazione}
{
    name=\glslink{serializzazione}{SERIALIZZAZIONE},
    text=Serializzazione,
    sort=serializzazione,
    description={In informatica, la \textit{serializzazione} è un processo per salvare un oggetto in un supporto di memorizzazione lineare (ad esempio, un file o un'area di memoria), o per trasmetterlo su una connessione di rete}
}

\newglossaryentry{cluster}
{
    name=\glslink{cluster}{CLUSTER},
    text=Cluster,
    sort=cluster,
    description={In informatica un computer \textit{cluster}, o più semplicemente un cluster (dall'inglese grappolo), è un insieme di computer connessi tra loro tramite una rete telematica. Scopo del cluster è distribuire un'elaborazione molto complessa tra i vari computer, aumentando la potenza di calcolo del sistema e/o garantendo una maggiore disponibilità di servizio, a prezzo di un maggior costo e complessità di gestione dell'infrastruttura: per essere risolto il problema che richiede molte elaborazioni viene infatti scomposto in sottoproblemi separati i quali vengono risolti ciascuno in parallelo}
}

\newglossaryentry{timestamp}
{
    name=\glslink{timestamp}{TIMESTAMP},
    text=Timestamp,
    sort=timestamp,
    description={Una marca temporale (\textit{timestamp}) è una sequenza di caratteri che rappresentano una data e/o un orario per accertare l'effettivo avvenimento di un certo evento. La data è di solito presentata in un formato compatibile, in modo che sia facile da comparare con un'altra per stabilirne l'ordine temporale}
}

\newglossaryentry{Apache Kafka}
{
    name=\glslink{Apache Kafka}{APACHE KAFKA},
    text=Apache Kafka,
    sort=apache kafka,
    description={\textit{Apache Kafka} è una piattaforma open source di stream processing scritta in Java e Scala e sviluppata dall'Apache Software Foundation. Il progetto mira a creare una piattaforma a bassa latenza ed alta velocità per la gestione di feed dati in tempo reale. Questo progetto viene usato principalmente per tutte le applicazioni di elaborazioni di stream di dati in tempo reale}
}

\newglossaryentry{pipeline}
{
    name=\glslink{pipeline}{PIPELINE},
    text=Pipeline,
    sort=pipeline,
    description={In informatica, il concetto di \textit{pipeline} (in inglese, tubatura — composta da più elementi collegati — o condotto) viene utilizzato per indicare un insieme di componenti software collegati tra loro in cascata, in modo che il risultato prodotto da uno degli elementi (output) sia l'ingresso di quello immediatamente successivo (input)}
}

\newglossaryentry{sqlg}
{
    name=\glslink{sql}{SQL},
    text=Structured Query Language,
    sort=sql,
    description={In informatica, \textit{SQL (Structured Query Language)} è un linguaggio standardizzato per database basati sul modello relazionale (RDBMS), progettato per:
    \begin{itemize}
    	\item{creare e modificare schemi di database;}
    	\item{inserire, modificare e gestire dati memorizzati;}
    	\item{interrogare i dati memorizzati;}
    	\item{creare e gestire strumenti di controllo e accesso ai dati.}
    \end{itemize}
    }
}

\newglossaryentry{query}
{
    name=\glslink{query}{QUERY},
    text=Query,
    sort=query,
    description={In informatica, il termine \textit{query} indica l'interrogazione di un database da parte di un utente. Il database (o base di dati) è in genere strutturato secondo il modello relazionale, che permette di compiere determinate operazioni sui dati (selezione, inserimento, cancellazione, aggiornamento, ecc.)}
}

\newglossaryentry{snapshot}
{
    name=\glslink{snapshot}{SNAPSHOT},
    text=Snapshot,
    sort=snapshot,
    description={Nei sistemi informatici, una \textit{snapshot} è un'istantanea dello stato di un sistema in un particolare momento, e può fare riferimento a una copia reale del sistema}
}

\newglossaryentry{pojog}
{
    name=\glslink{pojo}{POJO},
    text=POJO,
    sort=pojo,
    description={Nell'ingegneria del software, \textit{POJO} è un acronimo di \textit{Plain Old Java Object}. Il nome è usato per accentuare che un oggetto dato è un oggetto ordinario Java, cioè che non segue nessuno dei maggiori modelli, delle convenzioni, o dei \gls{framework} di oggetto Java}
}

