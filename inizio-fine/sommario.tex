% !TEX encoding = UTF-8
% !TEX TS-program = pdflatex
% !TEX root = ../tesi.tex

%**************************************************************
% Sommario
%**************************************************************
\cleardoublepage
\phantomsection
\pdfbookmark{Sommario}{Sommario}
\begingroup
\let\clearpage\relax
\let\cleardoublepage\relax
\let\cleardoublepage\relax

\chapter*{Sommario}

Il presente documento descrive il lavoro svolto durante il periodo di stage, della durata di circa trecento ore, dal laureando Niccolò Mantovani presso l'azienda Datasoil s.r.l.\\
L'obiettivo principale riguardava la creazione di una componente per il rilevamento di anomalie su risorse differenti, le quali fanno parte di impianti dislocati su un dato perimetro geografico e possono essere, per esempio, turbine eoliche oppure macchinari aziendali. Le risorse sono intese come distinte perché trattano informazioni provenienti da \textit{asset} diversi, come per esempio la rilevazione della temperatura fornita da un climatizzatore in relazione alla temperatura rilevata da un termostato digitale. I dati forniti, per essere analizzati, vengono raggruppati fra di loro per informazioni quali perimetro geografico, tipologia o categoria di risorsa.\\
L'elaborazione dei dati ha richiesto la modifica dell'operatore di aggregazione su una data finestra temporale e dell'operatore di rilevamento di anomalie sviluppati tramite il \textit{\textit{\gls{framework}}} \textit{Flink} utilizzando il linguaggio di programmazione \textit{Scala}.\\
In seguito si è reso necessario modificare le \textit{\gls{api}} di governo, sviluppate tramite il linguaggio di programmazione \textit{Go}, le quali permettono la configurazione da parte dell'utente di tale componente.\\
La componente realizzata durante il periodo di stage ha l'intento di permettere agli utenti di effettuare un'analisi degli eventi a livello d'insieme, raggruppando dati provenienti da risorse disgiunte, per garantire un livello informativo che può essere rilevante in numerosi ambiti.

%\vfill
%
%\selectlanguage{english}
%\pdfbookmark{Abstract}{Abstract}
%\chapter*{Abstract}
%
%\selectlanguage{italian}

\endgroup			

\vfill

