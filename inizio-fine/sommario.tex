% !TEX encoding = UTF-8
% !TEX TS-program = pdflatex
% !TEX root = ../tesi.tex

%**************************************************************
% Sommario
%**************************************************************
\cleardoublepage
\phantomsection
\pdfbookmark{Sommario}{Sommario}
\begingroup
\let\clearpage\relax
\let\cleardoublepage\relax
\let\cleardoublepage\relax

\chapter*{Sommario}

Il presente documento descrive il lavoro svolto durante il periodo di stage, della durata di circa trecento ore, dal laureando Niccolò Mantovani presso l'azienda Datasoil s.r.l.\\
L'obbiettivo principale riguardava la creazione di una componente per il rilevamento di anomalie su risorse differenti, raggruppate fra di loro, richiedendo la modifica dell'operatore di aggregazione su una data finestra temporale e dell'operatore di rilevamento di anomalie.\\
In seguito si è reso necessario modificare le \textit{\gls{api}} di governo le quali permettono la configurazione da parte dell'utente di tale componente.\\
Le tecnologie principali utilizzate per lo sviluppo del progetto nella sua totalità sono state il \textit{\textit{\gls{framework}}} \textit{Flink} e i linguaggi di programmazione \textit{Scala} e \textit{Go}.

%\vfill
%
%\selectlanguage{english}
%\pdfbookmark{Abstract}{Abstract}
%\chapter*{Abstract}
%
%\selectlanguage{italian}

\endgroup			

\vfill

