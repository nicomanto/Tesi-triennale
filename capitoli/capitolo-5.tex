% !TEX encoding = UTF-8
% !TEX TS-program = pdflatex
% !TEX root = ../tesi.tex

%**************************************************************
\chapter{Verifica e validazione}
\label{cap:verifica-validazione}
%**************************************************************


\intro{Il suddetto capitolo esplicita la fase di \textit{testing} relativa agli operatori e API descritte nella fase di sviluppo e codifica. Tale fase non è da considerare come successiva a quella della codifica del codice, ma parallela; infatti per ogni modifica apportate al codice si è deciso di valutarne il corretto funzionamento in modo immediato.}

\section{Classificazione dei test}
I test realizzati sono \textbf{\textit{test di unità}} e sono codificati come segue:
\begin{center}
\textbf{TU[Codice]}\\
\end{center}
dove \textbf{Codice} identifica un numero incrementale che distingue i vari test.

\section{Operatore Windowing}
Tale sezione elenca i test effettuati per garantire il corretto funzionamento dell'operatore \textit{Windowing} rispetto ai requisiti prefissati.

{
\centering
\begin{longtable}{L{3cm} L{8.5cm} L{2cm}}
\caption{Test di unità dell'operatore \textit{Windowing}}\\
\textbf{Identificativo} &
\textbf{Descrizione}&
\textbf{Esito}\\
\endhead
\hline

TU1 & L'operatore non deve aggiornare la propria configurazione se essa è eguale a quella nuova & Superato \\
\hline
TU2 & L'operatore aggiorna l'attuale configurazione in quanto la precedente ha una lunghezza dell'\textit{array DataFields} differente rispetto a quella nuova & Superato \\
\hline
TU3 & L'operatore aggiorna l'attuale configurazione in quanto la precedente ha un \textit{id} del gruppo, all'interno del campo \textit{assetFilters}, mancante nella nuova configurazione & Superato\\
\hline
TU4 & L'operatore aggiorna l'attuale configurazione in quanto la precedente ha un \textit{DataField} non più presente in quella nuova & Superato \\
\hline
TU5 & L'operatore aggiorna l'attuale configurazione in quanto la precedente ha una durata della \textit{window} differente da quella attuale & Superato\\
\hline
TU6 & L'operatore salta l'attuale finestra temporale avente durata uguale a zero. Viene verificato che vengano collezionati comunque i dati & Superato \\
\hline
TU7 & L'operatore non colleziona l'attuale evento in quanto non esiste una configurazione della \textit{window} & Superato \\
\hline
TU8 & L'operatore non colleziona l'attuale evento in quanto non contiene parametri su cui si può fare un'analisi (evento non arricchito) & Superato \\
\hline
TU9 & L'operatore configurato con la modalità \textit{barrier} crea correttamente un \textit{GroupedEvents} allineando eventi aggregati su se stessi. Tale aggregazione avviene correttamente sui campi all'interno di \textit{ctxDataField} & Superato \\
\hline
TU10 & L'operatore configurato con la modalità \textit{barrier} crea correttamente un \textit{GroupedEvents} allineando eventi aggregati su se stessi escludendo \textit{asset} che non contengono alcun \textit{DataField} & Superato \\
\hline
TU11 & L'operatore configurato con la modalità \textit{barrier} crea correttamente un \textit{GroupedEvents}, allineando eventi aggregati su se stessi, i quali mantengono i rispettivi dati che non devono essere aggregati. Inoltre gli eventi che non contenevano un determinato dato da aggregare, risultano non avere tale parametro settato con un valore di \textit{default} & Superato \\
\hline
TU12 & L'operatore configurato con la modalità \textit{barrier} crea correttamente un \textit{GroupedEvents} contenente un evento aggregato su se stesso secondo la modalità "\textit{somma}" & Superato \\
\hline
TU13 & L'operatore configurato con la modalità \textit{barrier} crea correttamente un \textit{GroupedEvents} contenente un evento aggregato su se stesso secondo la modalità "\textit{media}" & Superato \\
\hline
TU14 & L'operatore configurato con la modalità \textit{barrier} crea correttamente un \textit{GroupedEvents} contenente due eventi i quali sono aggregati su se stessi secondo la modalità "\textit{somma}" & Superato \\
\hline
TU15 & L'operatore configurato con la modalità \textit{barrier} crea correttamente un \textit{GroupedEvents} contenente due eventi i quali sono aggregati su se stessi secondo la modalità "\textit{media}" & Superato \\
\hline
TU16 & L'operatore configurato con la modalità \textit{barrier} crea correttamente l'aggregatore della futura finestra temporale con gli eventi in anticipo visti nell'attuale \textit{window} & Superato \\
\hline
TU17 & L'operatore configurato con la modalità \textit{barrier} colleziona correttamente due \textit{GroupedEvents}, cioè crea due gruppi differenti di eventi & Superato \\
\hline
TU18 & L'operatore configurato con la modalità \textit{barrier} gestisce correttamente la chiusura di tre finestre temporali & Superato \\
\hline
TU19 & L'operatore configurato con la modalità \textit{aggregation\_mean} non colleziona il \textit{GroupedEvents} poichè almeno un elemento durante l'aggregazione collettiva di \textit{asset} differenti non conteneva un particolare \textit{DataField} richiesto per l'aggregazione stessa & Superato \\
\hline
TU20 & L'operatore configurato con la modalità \textit{aggregation\_sum} crea correttamente un \textit{GroupedEvents} contenente un evento rappresentante l'aggregazione di \textit{asset} differenti & Superato \\
\hline
TU21 & L'operatore configurato con la modalità \textit{aggregation\_mean} crea correttamente un \textit{GroupedEvents} contenente un evento rappresentante l'aggregazione di \textit{asset} differenti & Superato \\
\hline
TU22 & L'operatore configurato con una modalità non prevista non crea nessun \textit{GroupedEvents} & Superato \\
\hline
\end{longtable}
}



\section{Operatore AlertCoProcess}
Tale sezione elenca i test effettuati per garantire il corretto funzionamento dell'operatore \textit{AlertCoProcess} rispetto ai requisiti prefissati

{
\centering
\begin{longtable}{L{3cm} L{8.5cm} L{2cm}}
\caption{Test di unità dell'operatore \textit{AlertCoProcess}}\\
\textbf{Identificativo} &
\textbf{Descrizione}&
\textbf{Esito}\\
\endhead
\hline

TU23 & L'operatore collega correttamente due \textit{asset} differenti alla stessa configurazione del \textit{detector} di tipo \textit{uniseas} & Superato\\
\hline
TU24 & L'operatore collega correttamente due \textit{asset} differenti a due configurazioni differenti di \textit{detector} di tipo \textit{uniseas} & Superato \\
\hline
TU25 &  L'operatore aggiorna correttamente la configurazione di un \textit{detector} di tipo \textit{uniseas} & Superato\\
\hline
TU26 & L'operatore svuota correttamente gli stati tramite una configurazione di tipo \textit{delete} su un \textit{detector} di tipo \textit{uniseas} & Superato \\
\hline
TU27 & L'operatore riesce a gestire correttamente l'errore relativo al caricamento fallito della configurazione di un \textit{detector} di tipo \textit{siblings} & Superato\\
\hline
TU28 & L'operatore ignora l'aggiornamento della configurazione perchè la nuova configurazione è di tipo "\textit{grouping}" ma il \textit{detector} configurato all'interno di essa è di tipo \textit{single input} & Superato \\
\hline
\end{longtable}
}

\section{API di configurazione}
Tale sezione elenca i test effettuati per garantire il corretto funzionamento delle \textit{\gls{api}} di governo per la gestione della configurazione della componente di \textit{anomaly detection}

{
\centering
\begin{longtable}{L{3cm} L{8.5cm} L{2cm}}
\caption{Test di unità \textit{API} di configurazione}\\
\textbf{Identificativo} &
\textbf{Descrizione}&
\textbf{Esito}\\
\endhead
\hline
TU29 & Viene verificata la correttezza dei messaggi d'errore forniti all'utente nel caso sia passata una configurazione non conforme alle attese, senza avere una configurazione preesistente  & Superato\\
\hline
TU30 & Viene verificata la correttezza dei messaggi d'errore forniti all'utente nel caso sia passata una configurazione non conforme alle attese, avendo una configurazione preesistente & Superato\\
\hline
TU31 & Viene verificata la correttezza della sostituzione degli \textit{id} forniti dall'utente, all'interno della configurazione, con gli \textit{id} generati dall'\textit{\gls{api}}, senza avere una configurazione preesistente & Superato\\
\hline
TU32 & Viene verificata la correttezza della sostituzione degli \textit{id} forniti dall'utente (sui nuovi gruppi), all'interno della configurazione, con gli \textit{id} generati dall'\textit{\gls{api}}, avendo una configurazione preesistente & Superato\\
\hline
\end{longtable}
}

\section{Tracciamento test di unità - ambito di sviluppo}
{
\centering
\begin{longtable}{L{5cm} L{8.5cm}}
\caption{Tracciamento test di unità - ambito di sviluppo}\\
\textbf{Identificativo test} &
\textbf{Ambito di sviluppo}\\
\endhead
\hline

TU1 \newline TU2 \newline TU3 \newline TU4 \newline TU5 \newline TU6 \newline TU7 \newline TU8 \newline TU9 \newline TU10 \newline TU11 \newline TU12 \newline TU13 \newline TU14 \newline TU15 \newline TU16 \newline TU17 \newline TU18 \newline TU19 \newline TU20 \newline TU21 \newline TU22 & Operatore \textit{Windowing} \\
\hline
TU23 \newline TU24 \newline TU25 \newline TU26 \newline TU27 \newline TU28 & Operatore \textit{AlertCoProcess} \\
\hline
TU29 \newline TU30 \newline TU31 \newline TU32 & \textit{\gls{api}} di configurazione\\
\hline
\end{longtable}
}


\section{Considerazioni}
I test elencati hanno tutti come esito \textbf{"Superato"}. Inoltre, per ogni modifica e funzionalità aggiuntiva apportata agli operatori \textit{Windowing} e \textit{AlertCoProcess}, sono stati creati test specifici per valutare sia la correttezza delle nuove funzionalità, sia la correttezza delle funzionalità preesistenti (\textbf{Test di regressione}). Per eseguire i test relativi agli operatori citati in precedenza, vengono sfruttate le funzionalità offerte da \textit{Flink} relative ai test di unità (\S\ref{sec:flink-testing}) ed il \textit{\textit{\gls{framework}}} \textit{ScalaTest} (\S\ref{sec:scala-test}).\\
Per i test relativi alle \textit{\gls{api}} di governo, viene sfruttata la funzionalità di \textit{testing} nativa nel linguaggio \textit{Go} (\S\ref{sec:go}).