% !TEX encoding = UTF-8
% !TEX TS-program = pdflatex
% !TEX root = ../tesi.tex

%**************************************************************
\chapter{Verifica e validazione}
\label{cap:verifica-validazione}
%**************************************************************


\intro{Il suddetto capitolo esplicita la fase di \textit{testing} relativa agli operatori descritti nella fase di sviluppo e codifica. Tale fase non è da considerare come successiva a quella della codifica del codice, ma parallela; infatti per ogni modifica apportate al codice si è deciso di valutarne il corretto funzionamento in modo immediato.}

\section{Operatore Windowing}
Tale sezione elenca i test effettuati per garantire il corretto funzionamento dell'operatore \textit{Windowing} rispetto ai requisiti prefissati

\subsection{Test di unità}
{
\centering
\begin{longtable}{L{3cm} L{8.5cm} L{2cm}}
\caption{Tabella riassuntiva test di unità dell'operatore \textit{Windowing}}\\
\textbf{Identificativo} &
\textbf{Descrizione}&
\textbf{Esito}\\
\endhead
\hline

TUW\textunderscore 1 & L'operatore non deve aggiornare la propria configurazione se essa è eguale a quella nuova & Superato \\
\hline
TUW\textunderscore 2 & L'operatore aggiorna l'attuale configurazione in quanto la precedente ha una lunghezza dell'\textit{array DataFields} differente rispetto a quella nuova & Superato \\
\hline
TUW\textunderscore 3 & L'operatore aggiorna l'attuale configurazione in quanto la precedente ha un \textit{id} del gruppo all'interno del campo \textit{assetFilters} mancante nella nuova configurazione & Superato\\
\hline
TUW\textunderscore 4 & L'operatore aggiorna l'attuale configurazione in quanto la precedente ha un \textit{DataField} non più presente in quella nuova & Superato \\
\hline
TUW\textunderscore 5 & L'operatore aggiorna l'attuale configurazione in quanto la precedente ha una durata della \textit{window} differente da quella attuale & Superato\\
\hline
TUW\textunderscore 6 & L'operatore salta l'attuale finestra temporale avente durata uguale a zero. Viene verificato che vengano collezionati comunque i dati & Superato \\
\hline
TUW\textunderscore 7 & L'operatore non colleziona l'attuale evento in quanto non esiste una configurazione della \textit{window} & Superato \\
\hline
TUW\textunderscore 8 & L'operatore non colleziona l'attuale evento in quanto non contiene parametri su cui si può fare un'analisi (evento non arricchito) & Superato \\
\hline
TUW\textunderscore 9 & L'operatore configurato con la modalità \textit{barrier} crea correttamente un \textit{GroupedEvents} allineando eventi aggregati su se stessi. Tale aggregazione avviene correttamente sui campi all'interno di \textit{ctxDataField} & Superato \\
\hline
TUW\textunderscore 10 & L'operatore configurato con la modalità \textit{barrier} crea correttamente un \textit{GroupedEvents} allineando eventi aggregati su se stessi escludendo \textit{asset} che non contengono alcun \textit{DataField} & Superato \\
\hline
TUW\textunderscore 11 & L'operatore configurato con la modalità \textit{barrier} crea correttamente un \textit{GroupedEvents} allineando eventi aggregati su se stessi i quali mantengono i rispettivi dati che non devono essere aggregati. Inoltre gli eventi che non contenevano un determinato dato su cui andava fatta l'aggregazione, non contengono quel determinato parametro settato con un valore di \textit{default} & Superato \\
\hline
TUW\textunderscore 12 & L'operatore configurato con la modalità \textit{barrier} crea correttamente un \textit{GroupedEvents} contenente un evento aggregato correttamente su se stesso secondo la modalità "\textit{somma}" & Superato \\
\hline
TUW\textunderscore 13 & L'operatore configurato con la modalità \textit{barrier} crea correttamente un \textit{GroupedEvents} contenente un evento aggregato correttamente su se stesso secondo la modalità "\textit{media}" & Superato \\
\hline
TUW\textunderscore 14 & L'operatore configurato con la modalità \textit{barrier} crea correttamente un \textit{GroupedEvents} contenente due eventi i quali sono aggregati correttamente su se stessi secondo la modalità "\textit{somma}" & Superato \\
\hline
TUW\textunderscore 15 & L'operatore configurato con la modalità \textit{barrier} crea correttamente un \textit{GroupedEvents} contenente due eventi i quali sono aggregati correttamente su se stessi secondo la modalità "\textit{media}" & Superato \\
\hline
TUW\textunderscore 16 & L'operatore configurato con la modalità \textit{barrier} crea correttamente l'aggregatore della futura finestra temporale con gli eventi in anticipo visti nell'attuale \textit{window} & Superato \\
\hline
TUW\textunderscore 17 & L'operatore configurato con la modalità \textit{barrier} colleziona correttamente due \textit{GroupedEvents}, cioè crea due gruppi differenti di eventi & Superato \\
\hline
TUW\textunderscore 18 & L'operatore configurato con la modalità \textit{barrier} gestisce correttamente la chiusura di tre finestre temporali & Superato \\
\hline
TUW\textunderscore 19 & L'operatore configurato con la modalità \textit{aggregator\textunderscore mean} non colleziona il \textit{GroupedEvents} poichè almeno un elemento durante l'aggregazione collettiva di \textit{asset} differenti non conteneva un particolare \textit{DataField} richiesto per l'aggregazione stessa & Superato \\
\hline
TUW\textunderscore 20 & L'operatore configurato con la modalità \textit{aggregator\textunderscore sum} crea correttamente un \textit{GroupedEvents} contenente un evento rappresentante l'aggregazione di \textit{asset} differenti & Superato \\
\hline
TUW\textunderscore 21 & L'operatore configurato con la modalità \textit{aggregator\textunderscore mean} crea correttamente un \textit{GroupedEvents} contenente un evento rappresentante l'aggregazione di \textit{asset} differenti & Superato \\
\hline
TUW\textunderscore 22 & L'operatore configurato con una modalità non prevista non crea nessun \textit{GroupedEvents} & Superato \\
\hline
\end{longtable}
}



\section{Operatore AlertCoProcess}
Tale sezione elenca i test effettuati per garantire il corretto funzionamento dell'operatore \textit{AlertCoProcess} rispetto ai requisiti prefissati

\subsection{Test di unità}
{
\centering
\begin{longtable}{L{3cm} L{8.5cm} L{2cm}}
\caption{Tabella riassuntiva test di unità dell'operatore \textit{AlertCoProcess}}\\
\textbf{Identificativo} &
\textbf{Descrizione}&
\textbf{Esito}\\
\endhead
\hline

TUA\textunderscore 1 & L'operatore collega correttamente due \textit{asset} differenti alla stessa configurazione del \textit{detector} di tipo \textit{uniseas} & Superato\\
\hline
TUA\textunderscore 2 & L'operatore collega correttamente due \textit{asset} differenti a due configurazioni differenti di \textit{detector} di tipo \textit{uniseas} & Superato \\
\hline
TUA\textunderscore 3 &  L'operatore aggiorna correttamente la configurazione di un \textit{detector} di tipo \textit{uniseas} & Superato\\
\hline
TUA\textunderscore 4 & L'operatore svuota correttamente gli stati tramite una configurazione di tipo \textit{delete} su un \textit{detector} di tipo \textit{uniseas} & Superato \\
\hline
TUA\textunderscore 5 & L'operatore riesce a gestire correttamente l'errore relativo al caricamento fallito della configurazione di un \textit{detector} di tipo \textit{siblings} & Superato\\
\hline
TUA\textunderscore 6 & L'operatore ignora l'aggiornamento della configurazione perchè la nuova configurazione è di tipo "\textit{grouping}" ma il \textit{detector} configurato all'interno di essa è di tipo \textit{single input} & Superato \\
\hline
\end{longtable}
}

\section{Considerazioni}
I test elencati hanno tutti come esito \textbf{"Superato"}, inoltre, per ogni modifica e funzionalità aggiuntiva apportata agli operatori, si è deciso di creare dei test per valutare sia la correttezza delle nuove funzionalità, sia la correttezza delle funzionalità preesistenti (\textbf{Test di regressione}). Per eseguire i test sono state sfruttate le funzionalità offerte da Flink relative ai test di unità (\S\ref{sec:flink-testing}) ed il \gls{framework} \textit{ScalaTest} (\S\ref{sec:scala-test})