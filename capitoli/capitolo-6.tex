% !TEX encoding = UTF-8
% !TEX TS-program = pdflatex
% !TEX root = ../tesi.tex

%**************************************************************
\chapter{Conclusioni}
\label{cap:conclusioni}
%**************************************************************

\intro{Tale capitolo espone i risultati ottenuti durante il periodo di stage relativamente al prodotto realizzato analizzando il risultato finale comprensivo degli obiettivi raggiunti, esponendo i possibili miglioramenti al prodotto finale e dando una valutazione soggettiva sull'intero percorso di stage. Infine vengono evidenziate le conoscenze acquisite.}

%**************************************************************
\section{Risultato finale}
Al termine della fase di sviluppo e successivamente alla validazione, la componente di \textit{anomaly detection} riesce ad analizzare \textbf{eventi differenti raggruppati} insieme emettendo, se necessario, \textit{alert} relativi al gruppo stesso se un'anomalia è stata rilevata. Pur avendo aggiunto l'analisi di anomalie gruppi di \textit{asset} differenti raggruppati, tale componente mantiene la preesistente funzionalità di rilevamento di anomalie su un \textit{asset} singolo, raggruppato oppure no su se stesso.\\

%**************************************************************
\section{Raggiungimento degli obiettivi}
Gli obiettivi prefissati ad inizio del percorso di stage sono risultati interamente soddisfatti, riuscendo a soddisfare sia quelli minimi che quelli massimi.\\
La componente di \textit{anomaly detection} realizzata soddisfa tutti i requisiti prefissati all'interno del capitolo Analisi dei Requisiti (\S\ref{cap:analisi-requisiti}). Oltre ai requisiti obbligatori risultano soddisfatti anche i requisiti desiderabili, considerati comunque importanti per dare valore aggiunto al prodotto realizzato, sopratutto perchè trattano principalmente la fase relativa ai \textit{test} considerata fondamentale per la validazione delle funzionalità principali della componente.

%**************************************************************
\section{Possibili miglioramenti ed estensioni}
Un possibile miglioramento alla componente di \textit{anomaly detection} è quello relativo all'aggiunta di nuovi tipi di \textit{detector}. Tale miglioramento garantirebbe l'analisi di anomalie più mirate rispetto agli attuali \textit{detector}, i quali sono stati pensati principalmente per operare con \textit{asset} eguali fra di loro.\\
Due possibili tipi di \textit{detector} potrebbero essere:
\begin{itemize}
	\item{\textbf{\textit{Unsupervised}}: \textit{detector} che lavora su un gruppo di risorse lavorando con un algoritmo di \textit{\gls{Apprendimento automatico}} di tipo \textit{\gls{Apprendimento non supervisionato}}. L'algoritmo richiederebbe un ricalcolo del modello nel caso di una nuova configurazione;}
	\item{\textbf{\textit{Rule based}}: \textit{detector} che lavora su un gruppo di risorse oppure su una risorsa singola in base alla propria configurazione e valutando l'andamento della/e risorsa/e tramite una regola definita dall'utente.}
\end{itemize}
In particolare, il \textit{detector} \textit{Rule based} permetterebbe all'utente di avere un maggiore controllo relativo al tipo di processamento dei dati, definendo da sé la logica utilizzata dal \textit{detector}.



%**************************************************************
\section{Conoscenze acquisite}
Il percorso di stage presso l'azienda \textit{Datasoil s.r.l.} mi ha dato conoscenze riguardo l'ambito di utilizzo di un tipo di analisi dei dati che è quello \textit{real-time}, il quale richiede una gestione in modo scalabile e ottimizzata al meglio per permettere un'analisi veloce e garantire il processo di flussi di dati molto grandi. Le conoscenze acquisite, inoltre, derivano anche dallo studio del \textit{\textit{\gls{framework}}} \textit{Flink} il quale fulcro di utilizzo riguarda il processamento di grandi quantità di mole di dati, sia \textit{batch} che \textit{real-time}. L'implementazione della componente di \textit{anomaly detection} riguardante gruppi di \textit{asset} differenti, inoltre, mi ha permesso di studiare e comprendere due tipi di linguaggio di programmazione mai utilizzati sino a questo momento, quali \textit{Go} e \textit{Scala}.\\
Questo periodo, infine, mi ha permesso di interfacciarmi con un vero e proprio ambito aziendale, permettendomi di comprendere il \textit{workflow} lavorativo e gestionale tipico di un'azienda di programmazione \textit{software}.

%**************************************************************
\section{Valutazione personale}
Lo stage è stata un'esperienza molto interessante e sopratutto formativa sia dal punto di vista della gestione del lavoro, sia riguardo le conoscenze tecniche acquisite. Ritengo che gli obiettivi prefissati siano stati adeguati per il carico di lavoro richiesto e, volenteroso di raggiungere anche gli obiettivi massimi quale il \textit{testing} di ciò che è stato creato, è stata richiesta una settimana aggiuntiva al periodo di stage prefissato.\\
Ritengo che l'ambito del progetto sia veramente innovativo, il quale diventerà, dal mio punto di vista, davvero importante negli anni a venire, per questo ringrazio l'azienda \textit{Datasoil s.r.l.} e il mio tutor \textit{Pietro De Caro} per avermi fatto conoscere e lavorare su un argomento davvero interessante ed innovativo. 
