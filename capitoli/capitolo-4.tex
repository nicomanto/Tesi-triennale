% !TEX encoding = UTF-8
% !TEX TS-program = pdflatex
% !TEX root = ../tesi.tex

%**************************************************************
\chapter{Progettazione e codifica}
\label{cap:progettazione-codifica}
%**************************************************************

\intro{Tale capitolo descrive in dettaglio la struttura del progetto, la progettazione e codifica delle nuove funzionalità, nonchè l'integrazione di esse con la struttura preesistente. La fase di progettazione e codifica non sono intese come attività prettamente distinte, infatti alcune analisi effettuate durante la progettazione iniziale antecedente alla codifica sono risultate non sufficienti a coprire numerose casistiche sorte durante lo sviluppo effettivo, per cui è stato necessario una riprogettazione mirata a risolvere le problematiche sorte.}

\section{Adattamento degli operatori preesistenti}
Durante l'analisi preventiva dei rischi relativa alle nuove funzionalità, sono sorti subito dei problemi relativi all'adattabilità degli operatori preesistenti con la nuove funzionalità create. Tali problematiche di adattabilità si possono riassumere in micro-aree:

\begin{itemize}
	\item{adattabilità della rappresentazione degli eventi tramite l'\textit{output} dell'\textbf{operatore \textit{Bumblebee}} definito come \textbf{\textit{BumblebeeOutput}}, che va a definire la trasformazione degli eventi "grezzi" in eventi arricchiti da informazioni necessarie per il loro processamento;}
	\item{adattabilità dell'\textbf{operatore \textit{Windowing}} che ora non deve solo trattare aggregazione di eventi ricevuti dallo stesso tipo di risorsa, ma deve gestire l'aggregazione e/o l'allineamento anche di risorse di tipo differente per permettere all'operatore \textit{AlertCoProcess} di eseguire un controllo a livello di insieme relativo a risorse differenti;}
	\item{adattabilità dell'\textbf{operatore \textit{AlertCoProcess}} per permettere ad esso di gestire e di controllare gruppi di eventi derivanti da risorse non più eguali fra di loro. Quindi di gestire il controllo relativo alla presenza di una anomalia relativa ad un gruppo di risorse. Inoltre, tale operatore, deve adattarsi anche alla presenze di due nuovi \textit{detector} di anomalie, quali \textbf{\textit{Unsupervised}} e \textbf{\textit{Rule based}}.}
\end{itemize}
	
Nello specifico si va ad analizzare le modifiche adottate riguardo le tre micro-aree descritte precedentemente.

%**************************************************************

\section{Operatore Bumblebee}\label{sec:bbout}
Per permettere la rappresentazione di gruppi di risorse differenti, viene cambiata la struttura dell'\textit{output} dell'operatore \textbf{\textit{Bumblebee}}, rappresentato inizialmente in un formato \gls{json} così formato:

\begin{minted}{js}
{
  "id": // id dell'evento, generato o trasformato dall'evento originale, in formato String,
  "pipeID": // id della pipeline, in formato String,
  "pipeVersion": //versione della pipeline, in formato String,
  "adapterID": //id dell'adattatore, in formato String,
  "tid": //id del cliente, in formato String,
  "assetID": //id della risorsa, in formato String,
  "ts": //timestamp dell'evento, in formato String,
  "asset": //informazioni relative alla risorsa dopo che è stata arricchita, in formato AssetInfo
  "data": //payload dei dati dell'evento trasformati in formato JSON,
  "ctxData": //dati relativi al contesto, come per esempio il tempo metereologico, in formato JSON,
  "meta": //metadati arricchiti dall'operatore di anomaly detection, in formato Option[EventMeta],
  "type": //tipo di BumblebeeOutput", in formato Option[String],
  "srcID": //id della fonte dell'evento in formato String
}
\end{minted}

Dopo l'opportuna modifica, la struttura rappresentate il \textbf{\textit{BumblebeeOutput}} risulta la seguente:

\begin{minted}{js}
{
  "id": // id dell'evento, generato o trasformato dall'evento originale, in formato String,
  "pipeID": // id della pipeline, in formato String,
  "pipeVersion": //versione della pipeline, in formato String,
  "adapterID": //id dell'adattatore, in formato String,
  "tid": //id del cliente, in formato String,
  "subject": {
	  "id": //id dell'oggetto, in formato String,
	  "type": //tipo dell'oggetto rapresentanto (asset,category,group), in formato String
  }
  "ts": //timestamp dell'evento, in formato String,
  "asset": //informazioni relative alla risorsa dopo che è stata arricchita, in formato AssetInfo
  "data": //payload dei dati dell'evento trasformati in formato JSON,
  "ctxData": //dati relativi al contesto, come per esempio il tempo metereologico, in formato JSON,
  "meta": //metadati arricchiti dall'operatore di anomaly detection, in formato Option[EventMeta],
  "type": //tipo di BumblebeeOutput", in formato Option[String],
  "srcID": //id della fonte dell'evento in formato String
}
\end{minted}
dove la rappresentazione è slegata da un \textit{asset} (risorsa) specifico. Per cui l'\textit{output} sarà adattato ad un \textbf{soggetto} generale, rappresentato da un proprio \textbf{\textit{id}} e da un suo \textbf{tipo}.
Tale struttura finale garantisce un corretto processamento del dato nel caso in cui si tratti di una risorsa singola oppure di un raggruppamento di quest'ultime.

%**************************************************************

\section{Operatore Windowing}
L'\textbf{operatore \textit{Windowing}} serve per rappresentare un determinato evento in una data finestra temporale. Inizialmente, tale operatore, trattava l'aggregazione solo di una determinata \textbf{risorsa} su se stessa, cioè \textbf{sommando} o \textbf{mediando} un determinato valore. Ora in tale operatore non solo permane il comportamento appena descritto, ma è presente anche la funzionalità che si occupa di allineare e/o aggregare fra di loro risorse di tipo differente, per permettere un processamento dei dati a livello collettivo. Per definire il comportamento della finestra temporale, è necessaria un'apposita \textbf{configurazione} che sarà iniettata durante l'elaborazione attiva dei dati, per permetterne la modifica del comportamento senza dover riavviare l'intero applicativo e creare uno stallo in esso.\\
L'operatore \textit{Windowing} è una \textbf{\textit{KeyedCoProcessFunction}}, così definita:
\begin{minted}{scala}
class Windowing(outputTagValues: OutputTag[AnomalyStepConfigurationCommand]) extends KeyedCoProcessFunction[String, BumblebeeOutput, AnomalyStepConfigurationCommand, GroupedEvents] {
	// ...
  }
\end{minted}
Dove in entrata avrà due tipi di flussi, uno relativo agli eventi da raggruppare di tipo \textit{BumblebeeOutput} (\S\ref{sec:pr1-windowing}) e uno relativo alla configurazione dell'operatore di tipo \textit{AnomalyStepConfiguration} (\S\ref{sec:pr2-windowing}). Tale operatore opera su flussi suddivisi per chiave definita come \textit{pipeID}.\\
Il metodo che verrà richiamato alla fine di una specifica finestra temporale è il metodo \textit{onTimer} (\S\ref{sec:on-timer-windowing}), il quale collezionerà gli eventi all'interno della \textit{window} appena conclusa e programmerà l'inizio della successiva finestra temporale.

\subsection{Stati dell'operatore}\label{sec:stati-windowing}
Per permettere una corretta gestione di entrambi i flussi, l'operatore \textit{Windowing} sfrutta i seguenti \textbf{stati} interni:
\begin{itemize}
		\item{\textbf{\textit{cfg}:} l'attuale configurazione della finestra temporale;}
		\item{\textbf{\textit{mapToAggregator}:} mappa che gestisce gli aggregatori (\S\ref{sec:aggregator}), i quali sono mappati per l'\textit{id} del gruppo;}
		\item{\textbf{\textit{mapToListFutureEvents}:} mappa che gestisce la lista degli eventi arrivati in anticipo, i quali sono mappati per l'\textit{id} del gruppo;}
		\item{\textbf{\textit{windowBeginTs}:} l'inizio dell'attuale finestra temporale;}
		\item{\textbf{\textit{scheduledTimer}:} \textit{timer} schedulato per collezionare gli eventi visti dalla \textit{window}, il quale coincide con la fine dell'attuale finestra temporale;}
		\item{\textbf{\textit{pipeStatus}:} mappa che gestisce la versione delle varie \gls{pipeline}, le quali sono mappate per l'\textit{id} della \gls{pipeline}.}
\end{itemize}


\subsection{Flusso degli eventi da raggruppare}\label{sec:pr1-windowing}
Tale flusso è gestito dal metodo \textit{processElement1}, il quale è un metodo ereditato dalla classe \textbf{KeyedCoProcessFunction} e sovrascritto dalla classe rappresentate l'operatore (\textit{Windowing}). Il flusso attuale dell'operatore prevede questi \textit{step}:

\begin{enumerate}
	\item{arrivo dell'evento di tipo \textit{BumblebeeOutput};}
	\item{controllo relativo a se la configurazione è presente, nel caso in cui non sia presente si saltano tutti i passaggi successivi e l'evento viene ignorato;}
	\item{per ogni gruppo contenuto all'interno del filtro di aggregazione della \textit{window} verrà fatto il controllo per verificare che quel determinato gruppo è filtrato per accettare quel determinato evento in entrata;}
\item{se l'evento è contenuto all'interno del filtro di un gruppo, viene controllata se la durata della \textit{window} è nulla (minore o uguale a zero), in questa casistica l'evento viene collezionato all'interno della struttura dati rappresentata dal \textit{GroupedEvents} (\S\ref{sec:ge}) senza essere aggregato o allineato con altri eventi. Il flusso relativo all'evento entrante termina;}
\item{se la durata della \textit{window} ha una durata non nulla (maggiore di zero), l'evento verrà inserito nell'attuale \textit{window} se il suo \gls{timestamp} è contenuto nei limiti di essa (inizio e fine). Se, invece, l'evento è antecedente all'inizio della \textit{window}, viene scartato. Infine, se l'evento è successivo alla fine della finestra temporale, verrà inserito all'interno dello stato \textbf{\textit{mapToListFutureEvents}} (\S\ref{sec:stati-windowing}). Tale evento verrà trattato nella successiva \textit{window}.}
\end{enumerate}

Per gestire l'inizio di una determinata nuova finestra temporale (quindi basata su una nuova \textbf{configurazione} per essa) verrà fatto un controllo per definire se l'evento entrante è il primo a partire da una nuova configurazione. In questo caso l'inizio della \textit{window} è eguale al \gls{timestamp} dell'evento entrante e la fine della \textit{window} è equivalente alla somma dell'inizio della finestra temporale con la sua durata (definita nella configurazione).
\subsection{Flusso di configurazione della window}\label{sec:pr2-windowing}
Tale flusso è gestito dal metodo \textit{processElement2}, il quale è un metodo ereditato dalla classe \textbf{KeyedCoProcessFunction} e sovrascritto dalla classe rappresentate l'operatore (\textit{Windowing}). Il flusso attuale dell'operatore prevede questi \textit{step}:

\begin{enumerate}
	\item{arrivo dell'evento riguardate la configurazione di tipo \textit{AnomalyStepConfiguration};}
	\item{controllo relativo a se la configurazione è di tipo "\textit{update}" o "\textit{delete}". In caso la configurazione non rappresenti nessuno dei due casi appena citati il flusso termina collezionando l'evento in un \textit{SideOutput}, cioè un coda differente rispetto a quella principale relativa ai \textit{GroupedEvents} (\S\ref{sec:ge}). Tale coda tratta gli eventi di tipo \textit{AnomalyStepConfiguration};}
	\item{Nel caso in cui la configurazione comprenda uno dei due tipi citati nel punto precedente, verrà fatto un controllo per verificare che la configurazione \textbf{relativa alla \textit{window}} (che è contenuta all'interno della configurazione di tipo \textit{AnomalyStepConfiguration}) sia presente e sia diversa da quella precedente, il che significa che la configurazione precedente è inesistente o che quella precedente si differenzia da quella attuale per almeno una di queste condizioni:
	\begin{itemize}
		\item{differente durata della finestra temporale;}
		\item{differente metodo di aggregazione;}
		\item{differente lista di gruppi su cui lavorerà la \textit{window};}
		\item{differenti parametri \textbf{relativi ai dati degli eventi} su cui si andrà a fare l'aggregazione;}
		\item{differenti \textbf{parametri del contesto} relativi agli eventi su cui si andrà a fare l'aggregazione.}
	\end{itemize}
	
Se la configurazione della \textit{window} è differente da quella precedente, vengono svuotati i relativi stati dell'operatore su cui opera la finestra temporale, quali \textbf{\textit{mapToAggregator}}, \textbf{\textit{mapToListFutureEvents}}, \textbf{\textit{windowBeginTs}} e \textbf{\textit{scheduledTimer}}. Di seguito viene aggiornato lo stato \textbf{\textit{cfg}} con i dati della finestra temporale nuova.\\
Se, invece, la nuova configurazione della finestra temporale è equivalente a quella precedente, non viene fatto nulla.\\
Infine viene aggiornato lo \textbf{\textit{pipeStatus}}.\\
Per una descrizione approfondita degli stati dell'operatore \textit{Windowing} fare riferimento alla sezione \S\ref{sec:stati-windowing}.
}
\item{Se invece la \textbf{configurazione della \textit{window}} non esiste, verrano svuotati tutti gli stati presenti nella sezione \S\ref{sec:stati-windowing}.}
\end{enumerate}

\subsection{OnTimer}\label{sec:on-timer-windowing}
Il metodo \textbf{\textit{onTimer}} è colui che gestisce la logica al termine di ogni finestra temporale. Tale funzione si occupa di creare un \textit{GroupedEvents} (\S\ref{sec:ge}) collezionando gli eventi visti secondo le regole definite dalla configurazione e ha il compito di configurare la prossima \textit{window}. Il metodo segue questi \textit{step}:
\begin{enumerate}
	\item{controllo relativo a se la configurazione della finestra temporale è settata, in caso negativo vengono saltati tutti gli \textit{step} successivi;}
	\item{in base alla modalità di aggregazione definita nella configurazione della finestra temporale, per ogni \textbf{\textit{id} del gruppo} contenuto all'interno dello stato \textbf{\textit{mapToAggregator}} verrà creato e collezionato un \textit{GroupedEvents} (\S\ref{sec:ge}) contenente gli eventi aggregati o allineati secondo la modalità definita, la quale può essere:
	\begin{itemize}
		\item{\textbf{\textit{barrier}:} in questa modalità gli eventi vengono \textbf{allineati} e il metodo che si occupa di questa funzionalità è il metodo \textit{selfAggregate} dell'aggregatore;}
		\item{\textbf{\textit{aggregation\textunderscore sum}:} in questa modalità gli eventi vengono raggruppati in uno unico, per cui verrà creato un nuovo \textit{BumblebeeOutput} che come parametri avrà il \gls{json} ritornato dal metodo \textit{groupAggregateDataFields} dell'aggregatore;}
		\item{\textbf{\textit{aggregation\textunderscore mean}:} in questa modalità gli eventi vengono raggruppati in uno unico, per cui verrà creato un nuovo \textit{BumblebeeOutput} che come parametri avrà il \gls{json} ritornato dal metodo \textit{groupAggregateDataFields} dell'aggregatore.}
	\end{itemize}
	Per la logica specifica di aggregazione vedere la sezione \S\ref{sec:aggregator};}
	\item{Creazione della nuova finestra temporale, cui l'inizio e la fine sono impostati rispettivamente con il \gls{timestamp} che ha attivato il metodo \textit{onTimer} e tale \gls{timestamp} sommato alla durata della \textit{window}. Di seguito vengono rimossi gli aggregatori all'interno dello stato \textbf{\textit{mapToAggregator}} che non contengono eventi inseriti nello stato \textbf{\textit{mapToListFutureEvents}} (nel caso in cui siano presenti degli eventi futuri, verrà istanziato un nuovo aggregatore contente già questi eventi). Infine viene svuotato lo stato \textbf{\textit{mapToListFutureEvents}}.
	}
\end{enumerate}

\subsection{GroupedEvents}\label{sec:ge}
L'\textit{output} dell'operatore, dapprima rappresentava solo una \textbf{risorsa singola}, ora invece tratterà un \textbf{gruppo}, il quale è identificato da un \textbf{\textit{id}}, da un \textbf{\gls{timestamp}} e da un \textbf{\textit{array}} di risorse di tipo \textit{BumblebeeOutput}.\\
Il \textit{GroupedEvents} può rappresentare logicamente due tipi logici di dato:
\begin{itemize}
	\item{\textbf{singolo elemento aggregato:} dove l'\textit{array} degli eventi è formato da un singolo elemento, l'\textit{id} del gruppo equivale all'\textit{id} definito dalla configurazione e il \textit{\gls{timestamp}} equivale a quello definito durante l'aggregazione effettiva;}
	\item{\textbf{allineamento:} dove l'\textit{array} degli eventi è formato da \textit{N} elementi, l'\textit{id} del gruppo equivale all'\textit{id} definito dalla configurazione e il \textit{\gls{timestamp}} equivale a quello definito durante l'aggregazione effettiva.}
\end{itemize}

\subsection{Aggregatore}\label{sec:aggregator}
L'aggregatore è la componente che si occupa di aggregare o allineare gli eventi all'interno della finestra temporale. L'operatore \textit{Windowing}, come spiegato nella sezione \S\ref{sec:stati-windowing} avrà uno stato che mapperà ogni \textit{id} del gruppo al suo specifico aggregatore. L'aggregatore in questione è di tipo \textit{BboutAggregator}, che è una classe contenente i seguenti parametri:

\begin{itemize}
	\item{\textbf{\textit{selfAggregatorMode}:} definisce il tipo di aggregazione che verrà fatto sugli \textit{asset} equivalenti fra di loro;}
	\item{\textbf{\textit{groupAccumulatorMode}:} definisce il tipo di aggregazione che verrà eseguito su \textit{asset} all'interno dello stesso gruppo;}
	\item{\textbf{\textit{acc}:} struttura che gestisce la mappatura fra un particolare \textit{asset} con una mappa che per chiave ha il nome di uno specifico parametro su cui si vuole andare a fare l'aggregazione e per valore ha la lista dei valori di quel parametro visti fino a quel determinato momento;}
	\item{\textbf{\textit{dataFields}:} \textit{array} contente i parametri degli eventi su cui si vuole fare l'aggregazione, decisi all'interno della \textbf{configurazione della \textit{window}};}
	\item{\textbf{\textit{ctxDataFields}:} \textit{array} contente i parametri del contesto degli eventi su cui si vuole fare l'aggregazione, decisi all'interno della \textbf{configurazione della \textit{window}}}
\end{itemize}

I metodi principali che gestiscono l'aggregazione a livello di \textbf{\textit{asset}} e a livello di \textbf{gruppo} sono:

\begin{itemize}
	\item{\textbf{\textit{add}:} tale metodo si occupa di salvare i valori dei parametri relativi agli \textit{asset} entranti nell'operatore \textit{Windowing}, dove tali parametri sono la concatenazione di \textit{dataFields} e \textit{ctxDataFields}. Per cui ad ogni chiamata di tale metodo verrà passato un particolare evento di tipo \textit{BumblebeeOutput}, tramite l'\textit{id} dell'evento viene estratta la mappa corrispondente ai parametri di \textit{dataFields} e \textit{ctxDataFields} e verrà aggiunto il valore del determinato parametro alla \textbf{lista dei valori} se esso è contenuto all'interno dell'evento.\\
Alla fine viene aggiornato il parametro \textit{acc} con la mappa aggiornata contenente la lista dei valori arricchita del nuovo valore, se trovato;}
	\item{\textbf{\textit{selfAggregate}:} tale metodo si occupa della vera e propria aggregazione degli \textit{asset} \textbf{su se stessi}, cioè per ogni \textit{asset} all'interno della mappa \textit{acc} (dove l'\textit{asset} è la chiave stessa) verranno aggregati i valori contenuti nel parametro \textit{dataFields} e \textit{ctxDataFields} secondo il metodo di aggregazione definito da \textit{selfAggregatorMode}.\\
	Se un determinato parametro ha la lista dei rispettivi valori minore o uguale a zero, vuol dire che esso non è appartenuto allo specifico \textit{BumblebeeOutput} che si sta aggregando, perciò tale parametro \textbf{non} verrà aggiunto con un valore di default durante l'aggregazione.\\
	Come tipo di ritorno tale metodo ritornerà un \textit{array} di \textit{BumbleebeeOutput} dove ogni \textit{asset} avrà i parametri interni a \textit{dataFields} e \textit{ctxDataFields} aggregati secondo il metodo definito e i parametri non interni ai due \textit{array} appena citati valorizzati con l'ultimo elemento visto di quel parametro. Se un determinato \textbf{asset} non contiene nessuno \textit{dataField}, non viene aggiunto all'\textit{array} ritornato;}
	\item{\textbf{\textit{groupAggregateDataFields}:} tale metodo si occupa di aggregare i parametri all'interno di \textit{asset} diversi fra di loro aggregati precedentemente su se stessi tramite il metodo \textit{selfAggregate}. Per ogni parametro contenuto nell'\textit{array} all'interno di \textit{dataFields} viene fatta l'aggregazione dei valori di quel determinato parametro all'interno dei \textit{BumblebeeOutput} ritornati dal metodo \textit{selfAggregate}. Il metodo ritorna un \gls{json} contenente l'aggregazione dei parametri contenuti in \textit{dataFields} secondo il metodo definito in \textit{groupAccumulatorMode}. Se un determinato parametro non è contenuto in almeno un \textit{BumblebeeOutput}, viene ritornato un \gls{json} vuoto, per permettere all'operatore di \textit{Windowing} di poter saltare la collezione del \textit{groupedEvents} che verrebbe creato.}
\end{itemize}



%**************************************************************
\section{Operatore AlertCoProcess}
L'operatore \textbf{\textit{AlertCoProcess}} è colui che si occupa di sollevare segnalazioni nel caso sia presente un'anomalia riguardo l'evento entrante di tipo \textit{GroupedEvents} (\S\ref{sec:ge}). Dapprima l'\textit{AlertCoProcess} lavorava solo su un tipo di evento legato ad una risorsa singola, mappando l'\textit{id} del \textit{detector} ad un'altra mappa dove la chiave è rappresentata dall'\textit{id} della risorsa e il valore è rappresentato dal vero e proprio \textit{detector}.\\
Ora, invece, la struttura è basata sull'\textit{id} del gruppo, per cui è stata creata un'ulteriore mappa fra \textit{id} del gruppo e \textit{id} del \textit{detector}, inoltre la chiave della mappa interna non è più rappresentata da un'\textit{id} della risorsa, ma da un \textit{id} del soggetto dopo la modifica di \textit{Bumblebee} (\S\ref{sec:bbout}).\\
L'operatore \textit{AlertCoProcess} è una \textbf{\textit{KeyedCoProcessFunction}}, così definita:
\begin{minted}{scala}
class AlertCoProcess (inspectionTag: OutputTag[MongoInsertDocument]) extends KeyedCoProcessFunction[String, GroupedEvents, AnomalyStepConfigurationCommand, (String, BumblebeeOutput)] {
	// ...
  }
\end{minted}
Dove in entrata avrà due tipi di flussi, uno relativo agli eventi su cui controllare se viene rilevata un'anomalia di tipo \textit{GroupedEvents} (\S\ref{sec:pr1-alertcoprocess}) e uno relativo alla configurazione dell'operatore di tipo \textit{AnomalyStepConfiguration} (\S\ref{sec:pr2-alertcoprocess}). Tale operatore opera su flussi suddivisi per chiave definita come \textit{pipeID}.\\
L'operatore lavora su un singolo tipo di \textit{detector}, il quale è legato indirettamente al \textit{pipeID}. Anche se il tipo di \textit{detector} è lo stesso, la configurazione dei parametri utilizzati da esso può cambiare a seconda del tipo gruppo che viene definito all'interno del \textit{GroupedEvents} entrante.


\subsection{Stati dell'operatore}\label{sec:stati-alertcoprocess}
Per permettere una corretta gestione di entrambi i flussi, l'operatore \textit{AlertCoProcess} sfrutta i seguenti \textbf{stati} interni:
\begin{itemize}
		\item{\textbf{\textit{isGroupingDetector}:} valore \textit{booleano} che identifica se il \textit{detector} dell'operatore tratta gli eventi come gruppo o no;}
		\item{\textbf{\textit{tid}:} identifica il cliente all'interno dell'\textit{AlertCoProcess};}
		\item{\textbf{\textit{mapToDetectorID}:} mappa che gestisce la correlazione fra l'\textit{id} del gruppo (chiave) e l'\textit{id} del \textit{detector} (valore);}
		\item{\textbf{\textit{mapToDetector}:} mappa che gestisce la correlazione fra il l'\textit{id} del \textit{detector} (chiave) e un ulteriore mappa (valore) che lega l'\textit{id} di un \textbf{soggetto}/\textbf{gruppo} con il suo vero e proprio \textit{detector};}
		\item{\textbf{\textit{detectorsConfigs}:} l'attuale configurazione del tipo di \textit{detector};}
		\item{\textbf{\textit{pipeStatus}:} mappa che gestisce la versione delle varie \gls{pipeline}, le quali sono mappate per l'\textit{id} della \gls{pipeline}.}
\end{itemize}

\subsection{Flusso degli eventi da controllare}\label{sec:pr1-alertcoprocess}
Tale flusso è gestito dal metodo \textit{processElement1}, il quale è un metodo ereditato dalla classe \textbf{KeyedCoProcessFunction} e sovrascritto dalla classe rappresentate l'operatore (\textit{AlertCoProcess}). Il flusso attuale dell'operatore prevede questi \textit{step}:
\begin{enumerate}
	\item{arrivo dell'evento di tipo \textit{GroupedEvents};}
	\item{viene estratta la mappa dei \textit{detector} tramite l'\textit{id} del \textit{detector}, estratto rispettivamente tramite l'\textit{id} del \textit{gruppo}. Ovviamente vengono effettuati dei controlli per verificare l'esistenza delle chiavi all'interno delle due mappe;}
	\item{se lo stato \textbf{\textit{isGroupingDetector}} ha come valore \textit{true}, vuol dire che l'anomalia, se presente, deve essere sollevata sul gruppo. Per cui viene estratto l'unico \textit{detector} (tramite l'\textit{id} del gruppo) dalla mappa estratta dallo \textit{step} precedente. Se il \textit{detector} è presente si utilizza quello preesistente, altrimenti ne viene creato uno nuovo. Infine viene creato un \textit{BumblebeeOutput} rappresentativo del gruppo su cui si sollevare la possibile anomalia rilevata dal \textit{detector};}
	\item{se lo stato \textbf{\textit{isGroupingDetector}}, invece, ha come valore \textit{false}, vuol dire che l'anomalia deve essere alzata su un singolo evento. Per cui per ogni evento interno all'\textit{array} degli eventi contenuto nell'oggetto \textit{GroupedEvents}, verrà estratto il \textit{detector} contenuto all'interno della mappa estratta nel punto 2. Su questo \textit{detector} viene controllato se è di tipo \textit{single input} (accetta un solo evento in entrata) oppure \textit{multiple input} (accetta un \textit{array} di eventi in entrata). A seconda dei due casi appena citati verrà passato o il singolo evento da cui si è estratto lo specifico \textit{detector} oppure tutto l'\textit{array} degli eventi. Nonostante ciò la possibile anomalia verrà rilevata sull'evento a cui è mappato il \textit{detector}.}
\end{enumerate}

\subsection{Flusso di configurazione dell'AlertCoProcess}\label{sec:pr2-alertcoprocess}
Tale flusso è gestito dal metodo \textit{processElement2}, il quale è un metodo ereditato dalla classe \textbf{KeyedCoProcessFunction} e sovrascritto dalla classe rappresentate l'operatore (\textit{AlertCoProcess}). Il flusso attuale dell'operatore prevede questi \textit{step}:

\begin{enumerate}
	\item{arrivo dell'evento di tipo \textit{AnomalyStepConfigurationCommand};}
	\item{la configurazione può essere di tre tipi differenti:
	\begin{itemize}
		\item{\textbf{\textit{update}:} in questo caso è richiesto l'aggiornamento dei \textit{detector}. Per ogni \textit{detector configuration} contenuto all'interno della nuova configurazione viene estratto l'\textit{id} del \textit{detector}, con il quale viene verificato se esso è presente all'interno dello stato \textbf{\textit{mapToDetectorID}}. Nel caso in cui l'\textit{id} del \textit{detector} sia contenuto all'interno della mappa verrà fatto l'accesso alla lista relativa di tutti i veri e propri \textit{detector} collegati a quel determinato \textit{id}. Per cui ogni \textit{detector} verrà aggiornato secondo la nuova configurazione fornita. La modalità di aggiornamento del \textit{detector} è specificata nella sezione \S\ref{sec:detector-alertcoprocess}. Nel caso, invece, in cui l'\textit{id} del \textit{detector} non sia contenuto all'interno della mappa verrà creata una mappa vuota che verrà arricchita nella funzione \textit{processElement1} (\S\ref{sec:pr1-alertcoprocess}). Infine vengono aggiornati i rispettivi stati che gestiscono la mappatura dei \textit{detector} (\textbf{\textit{mapToDetectorID}} e \textbf{\textit{mapToDetector}}) rimuovendo, se necessario, i \textit{detector} non più richiesti;}
		\item{\textbf{\textit{delete}:} questo tipi di configurazione si occupa di svuotare gli stati relativi alla gestione dei \textit{detector}, quali \textbf{\textit{mapToDetectorID}} e \textbf{\textit{mapToDetector}}. Infine vengono svuotati gli stati \textbf{\textit{detectorsConfigs}} e \textbf{\textit{pipeStatus}};}
		\item{\textbf{\textit{inspect}:} tale configurazione permettere di ispezionare il comportamento effettivo dei \textit{detector} collezionando delle informazioni mirate.}
	\end{itemize}}
\end{enumerate}


\subsection{Lista dei detector}\label{sec:detector-alertcoprocess}
Di seguito la lista rappresentante i \textit{detector} utilizzati dall'\textit{AlertCoProcess}:
\begin{itemize}
	\item{\textbf{\textit{Uniseas}}: \textit{detector} che lavora singolarmente su un tipo di risorsa. Il suo aggiornamento non richiede un ricalcolo del modello non usando un algoritmo di \gls{Apprendimento automatico};}
	\item{\textbf{\textit{Crosscorel}}: \textit{detector} che lavora singolarmente su un tipo di risorsa e su un evento esterno ad essa (come per esempio il tempo metereologico). Il suo aggiornamento non richiede un ricalcolo del modello non usando un algoritmo di \gls{Apprendimento automatico};}
	\item{\textbf{\textit{Siblings}}: \textit{detector} che lavora singolarmente su un tipo risorsa valutando il suo andamento su \textit{N} risorse uguali ad essa. Utilizza un algoritmo di \gls{Apprendimento automatico} di tipo \gls{Apprendimento supervisionato}. A differenza dei precedenti \textit{detector}, l'algoritmo richiede un ricalcolo del modello nel caso di una nuova configurazione;}
	\item{\textbf{\textit{Unsupervised}}: \textit{detector} che lavora su un gruppo di risorse lavorando con un algoritmo di \gls{Apprendimento automatico} di tipo \gls{Apprendimento non supervisionato}. L'algoritmo richiede un ricalcolo del modello nel caso di una nuova configurazione;}
	\item{\textbf{\textit{Rule based}}: \textit{detector} che lavora su un gruppo di risorse oppure su una risorsa singola in base alla propria configurazione e valutando l'andamento della/e risorsa/e tramite una regola definita dall'utente.}
\end{itemize}

%**************************************************************

\section{API della configurazione}
Dopo le modifiche apportate agli operatori precedenti è richiesta una nuova definizione delle \gls{api} di governo che gestiscono i messaggi che vanno a definire la configurazione dell'\textit{anomaly detection}, cioè la configurazione dell'operatore \textit{Windowing} e dell'operatore \textit{AlertCoProcess}. La nuova definizione delle \gls{api} consiste in:

\begin{itemize}
	\item{definizione della nuova struttura del messaggio in formato \gls{json} contenente le configurazioni decise dall'utente;}
	\item{controllo relativo alla correttezza e consistenza dei dati passati dall'utente.}
\end{itemize}
Per semplicità la definizione del messaggio fornito dall'\gls{api} viene suddiviso e analizzato in sezioni differenti per esplicitare al meglio la parte relativa all'operatore \textit{Windowing} e quella relativa ai \textit{detector} utilizzati all'interno dell'operatore \textit{AlertCoProcess}.




\subsection{JSON di configurazione dell'operatore Windowing}\label{sec:json-windowing}
La definizione della configurazione relativa all'operatore \textit{Windowing} è così rappresentata:

\begin{minted}{js}
"windowing":{
    "duration": 10, //durata della finestra temporale in minuti,
    "aggregationMethod": //tipo di aggregazione per asset eguali fra di loro (sum|mean), in formato String,
    "windowingMode": //modalità di raggruppamento a gruppi (barrier,aggregation_sum,aggregation_mean), in formato String,
    "dataFields": //parametri caratterizzante gli asset
    [
      {
        "name": // nome del parametro in formato String,
        "kind": // tipo del parametro (continuous|categorical), in formato String
      }
    ],
    "ctxDataFields": //parametri del contesto caratterizzante gli asset 
    [
      {
        "name":  // nome del parametro in formato String,
        "kind": // tipo del parametro (continuous|categorical), in formato String
      }
    ],
    "assetFilters": { // filtri da applicare all'interno della finestra temporale
        "groupID": {
            "id": //id identificativo del gruppo (uguale al groupID), in formato String,
            "filterGeos": // id relativi alla posizione geografica di un asset, in formato Array[String],
            "filterBfuncs": //id relativi ad una funzione, in formato Array[String]
            "filterCats": //id relativi alla categoria di un asset, in formato Array[String]
            "filteredAssetsIds": //id relativi agli asset, in formato Array[String]
        } 
    }
}
\end{minted}
Le principali informazioni aggiuntive introdotte nel formato \gls{json} relativo alla configurazione della finestra temporale sono:
\begin{itemize}
	\item{\textbf{\textit{windowingMode}:} definizione della modalità di raggruppamento su gruppi;}
	\item{\textbf{\textit{assetFilters}}: ora i filtri da applicare ad una finestra temporale sono mappati per l'\textit{id} del gruppo. Così facendo si permette di avere filtraggi differenti per gruppi differenti.}
\end{itemize} 


\subsection{JSON di configurazione dei detector}\label{sec:json-detectors}
La definizione della configurazione relativa ai \textit{detector} è così rappresentata:
\begin{minted}{js}
"detectors": [
   {
    "detectorID": // id del detector, in formato String, 
    "detectorType": // tipo del detector (siblings|uniseas|ctx_ad_cross), in formato String, 
    "parameters": {
		// parametri relativi al tipo di detector    
    },
    "assetFilters":{
      "id": //id identificativo del gruppo (uguale al groupID), in formato String,
      "filterGeos": // id relativi alla posizione geografica di un asset, in formato Array[String],
      "filterBfuncs": //id relativi ad una funzione, in formato Array[String]
      "filterCats": //id relativi alla categoria di un asset, in formato Array[String]
      "filteredAssetsIds": //id relativi agli asset, in formato Array[String]
    }
   }
  ]
}
\end{minted}
La principale informazione aggiuntiva introdotta nel formato \gls{json} relativo alla configurazione dei \textit{detector} è rappresentata dall'\textit{id} relativo al gruppo introdotto all'interno di \textit{assetFilters}. Tale \textit{id} permette una corretta mappatura del gruppo con l'\textit{id} del \textit{detector} all'interno dell'operatore \textit{AlertCoProcess}.\\
Si è deciso di non rappresentare la struttura relativa ai parametri che identificano il \textit{detector} (rappresentanti dal campo \textit{parameters}) perchè essi rimangono invariati anche dopo l'aggiunta delle nuove funzionalità e sono scollegate da essa.

\subsection{JSON configurazione Anomaly Detection}
Di seguito viene rappresentato il messaggio \gls{json} di configurazione globale fornito dall'\gls{api} che si occupa di definire la configurazione decisa dall'utente:

\begin{minted}{js}
{
  "ctype": // tipo della configurazione cui dipende la deserializzazione, in formato String,
  "tid": //id del cliente, in formato String, 
  "pipeID": // id della pipeline, in formato String, 
  "pipeVersion": //versione della pipeline, in formato String,
  "outputTopic": // tipo di topic dell'output, in formato String,
  "isGrouping": // rappresenta se l'anomalia deve essere sollevata su un gruppo, in formato Boolean,
  "detectors": // configurazione dei detectors, in fomato Array,
  "windowing": // configurazione della finestra temporale
 }
\end{minted}
dove:
\begin{itemize}
	\item{\textbf{\textit{detectors}}: rappresenta un \textit{array} di configurazioni come descritto nella sezione \S\ref{sec:json-detectors}};
	\item{\textbf{\textit{windowing}}: rappresenta la configurazione della finestra temporale come descritto nella sezione \S\ref{sec:json-windowing}.}
\end{itemize}

Rispetto alla configurazione iniziale è stato introdotto il campo \textit{isGrouping}, che definisce se l'anomalia deve essere sollevata su un gruppo oppure no.

\subsection{Correttezza e consistenza dei dati forniti dall'utente}
A fronte della modifica del messaggio di configurazione fornito alla componente di \textit{anomaly detection}, serve garantire che i dati forniti dall'utente che tratta tale configurazione siano corretti e consistenti fra di loro.\\
Di seguito verrà descritto il controllo effettuato per garantire che la configurazione fornita sia corretta, sia secondo la sua configurazione, sia secondo la configurazione preesistente nel caso si tratti di una modifica di essa.\\
Dopo la ricezione dei dati forniti dall'utente viene effettuato il seguente controllo:
\begin{itemize}
	\item{\textbf{verifica della consistenza della nuova configurazione:}
		\begin{enumerate}
			\item{si verifica che la lunghezza della mappa \textit{windowing.assetFilters} sia eguale alla lunghezza dell'\textit{array} \textit{detectors};}
			\item{per ogni elemento all'interno della mappa \textit{windowing.assetFilters} si verifica che la chiave dell'elemento (\textit{groupID}) sia corrispondente al campo \textit{id} contenuto dall'elemento stesso;}
			\item{per ogni elemento all'interno della mappa \textit{windowing.assetFilters} si verifica che tale elemento sia contenuto all'interno dell'\textit{array} \textit{detectors}.}
		\end{enumerate}}
	\item{\textbf{verifica della consistenze della nuova configurazione relativamente a quelle preesistente:} per ogni chiave della mappa \textit{windowing.assetFilters}, preesistente nella vecchia configurazione, viene verificato che il valore dell'\textit{assetFilter} contenuto sia rimasto invariato rispetto alla precedente configurazione.}
\end{itemize}

Per ogni nuovo gruppo viene sostituito l'\textit{id} fornito dall'utente, richiesto per identificare la correlazione fra l'\textit{assetFilter} del gruppo contenuto all'interno della mappa \textit{windowing.assetFilters} e il medesimo \textit{assetFilter} contenuto nell'\textit{array} \textit{detectors}, con un \textit{id} appositamente creato dall'\gls{api}.\\
Per ogni controllo appena elencato viene fornito un messaggio d'errore all'utente nel caso non venga superato tale controllo, questo per comunicare all'utente che tale configurazione non è corretta.


