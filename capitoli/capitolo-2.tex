% !TEX encoding = UTF-8
% !TEX TS-program = pdflatex
% !TEX root = ../tesi.tex

%**************************************************************
\chapter{Strumenti e tecnologie utilizzate}
\label{cap:strumenti-tecnologie}

%**************************************************************

\intro{Tale capitolo racchiude gli strumenti utilizzati e le tecnologie impiegate durante la realizzazione di \myTitle. In particolare verra analizzato il dominio di utilizzo di esse e le varie meotodologie utilizzate.}\\

%**************************************************************
\section{Apache Flink}
Apache Flink è un \textit{framework}\ped{G} di \textit{data processing engine}\ped{G} per l'elaborazione e processamento di dati in modo distribuito.
La forza di Flink è quella di trattare dati in differenti formati, oppurtuni per il dominio applicativo più consono per l'utente, infatti i dati possono essere di tipo \textit{stateful}\ped{G} che \textit{stateless}\ped{G} e tali flussi possono essere illimitati (\textit{unbounded}\ped{G}) o limitati (\textit{bounded}\ped{G}). Flink è stato progettato per funzionare in tutti i comuni ambienti \textit{cluster\ped{G}}.

\subsection{Tipi di flussi}
\subsubsection{Flussi illimitati (unbounded\ped{G})}
\subsubsection{Flussi limitati (bounded\ped{G})}

\subsection{Stateful\ped{G} e Stateless\ped{G}}

\subsubsection{Savepoint\ped{G} e Checkpoint\ped{G}}

\subsubsection{Serializzazione\ped{G}}

\subsection{Scalabilità}

