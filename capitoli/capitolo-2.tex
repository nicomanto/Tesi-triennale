% !TEX encoding = UTF-8
% !TEX TS-program = pdflatex
% !TEX root = ../tesi.tex

%**************************************************************
\chapter{Strumenti e tecnologie utilizzate}
\label{cap:strumenti-tecnologie}

%**************************************************************

\intro{Tale capitolo racchiude gli strumenti utilizzati e le tecnologie impiegate durante la realizzazione di \myTitle. In particolare verra analizzato il dominio di utilizzo di esse e le varie meotodologie utilizzate.}\\

%**************************************************************
\section{Apache Flink}
Apache Flink è un \gls{framework} di \textit{data processing engine} per l'elaborazione e processamento di dati in modo distribuito.
La forza di Flink è quella di trattare dati in differenti formati, opportuni per il dominio applicativo più consono per l'utente, infatti i dati possono essere di tipo \gls{stateful} che \gls{stateless} e tali flussi possono essere illimitati (\gls{unbounded streams}) o limitati (\gls{bounded streams}). Flink è stato progettato per funzionare in tutti i comuni ambienti \gls{cluster}.

\subsection{Datastream API}
	\subsubsection{Flussi illimitati (unbounded)}
	\subsubsection{Flussi limitati (bounded)}

\subsection{Table API}

	\subsection{Operatori}
		\subsubsection{FlatMap}
		\subsubsection{CoFlatMap}
		\subsubsection{Window}
		\subsubsection{Timer}


\subsection{Alta disponibilità e recovery}

	\subsubsection{Stateful e Stateless}

	\subsubsection{Savepoint e Checkpoint}

\subsection{Serializzazione}

	\subsubsection{Avro}

	\subsubsection{Kryo}


\subsection{Scalabilità}


\section{Scala}

\section{Amazon Kinesis}

