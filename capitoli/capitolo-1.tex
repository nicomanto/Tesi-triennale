% !TEX encoding = UTF-8
% !TEX TS-program = pdflatex
% !TEX root = ../tesi.tex

%**************************************************************
\chapter{Introduzione}
\label{cap:introduzione}
%**************************************************************

Introduzione al contesto applicativo.\\

\noindent Esempio di utilizzo di un termine nel glossario \\
\gls{api}. \\

\noindent Esempio di citazione in linea \\
\cite{site:agile-manifesto}. \\

\noindent Esempio di citazione nel pie' di pagina \\
citazione\footcite{womak:lean-thinking} \\

%**************************************************************
\section{L'azienda}

Descrizione dell'azienda.

%**************************************************************
\section{Il progetto}

Introduzione all'idea dello stage.

\subsection{Descrizione}

\subsection{Principali problematiche e relative soluzioni}

Durante la fase di analisi iniziale sono stati individuati alcuni possibili rischi a cui si potrà andare incontro.
Si è quindi proceduto ad elaborare delle possibili soluzioni per far fronte a tali rischi.\\

\begin{risk}{Adattabilità operatore Bumblebee}
    \riskdescription{l'operatore preesistente deve riuscire a creare un output informativo che può trattare gruppi di risorse (eguali o diverse fra di loro) o risorse singole}
    \risksolution{Modifica della struttura descrittiva dell'output per garantire la gestione di entrambi i casi a livelli successivi di \gls{pipeline}}
    \label{risk:bumblebeeOutput} 
\end{risk}

\begin{risk}{Adattabilità operatore di windowing}
    \riskdescription{l'operatore preesistente deve riuscire a gestire contemporaneamente l'aggregazione e l'allineamento di gruppi di risorse (eguali o diverse fra di loro)}
    \risksolution{adattare tale operatore alle nuove esigenze garantendo la retrocompatibilità con il funzionamento precedente}
    \label{risk:windowing} 
\end{risk}

\begin{risk}{Adattabilità operatore Anomaly Detector}
    \riskdescription{l'operatore preesistente deve riuscire ad alzare anomalie su gruppi di risorse differenti fra di loro e adattarsi all'aggiunta di nuovi detector}
    \risksolution{adattare tale operatore alle nuove esigenze garantendo la retrocompatibilità con il funzionamento precedente}
    \label{risk:anomalyDetector} 
\end{risk}

\subsection{Pianificazione}

\subsection{Obiettivi}

%**************************************************************

\section{Il prodotto finale}


%**************************************************************
\section{Organizzazione del testo}

\begin{description}
    \item[{\hyperref[cap:processi-metodologie]{Il secondo capitolo}}] descrive ...
    
    \item[{\hyperref[cap:descrizione-stage]{Il terzo capitolo}}] approfondisce ...
    
    \item[{\hyperref[cap:analisi-requisiti]{Il quarto capitolo}}] approfondisce ...
    
    \item[{\hyperref[cap:progettazione-codifica]{Il quinto capitolo}}] approfondisce ...
    
    \item[{\hyperref[cap:verifica-validazione]{Il sesto capitolo}}] approfondisce ...
    
    \item[{\hyperref[cap:conclusioni]{Nel settimo capitolo}}] descrive ...
\end{description}

Riguardo la stesura del testo, relativamente al documento sono state adottate le seguenti convenzioni tipografiche:
\begin{itemize}
	\item gli acronimi, le abbreviazioni e i termini ambigui o di uso non comune menzionati vengono definiti nel glossario, situato alla fine del presente documento;
	\item per la prima occorrenza dei termini riportati nel glossario viene utilizzata la seguente nomenclatura: \emph{parola}\glsfirstoccur;
	\item i termini in lingua straniera o facenti parti del gergo tecnico sono evidenziati con il carattere \emph{corsivo}.
\end{itemize}